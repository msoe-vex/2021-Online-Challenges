\documentclass[]{formalLabReport}

\usepackage{graphicx}

\usepackage{tikz}

%%%%%%%%%%%%%%%%%%%%%%%%%%%%%%%%%%%%%%%%%%%%%%%%%%%%%%%%%%%%%%%%%%%%%%
% LaTeX Overlay Generator - Annotated Figures v0.0.1
% Created with http://ff.cx/latex-overlay-generator/
% If this generator saves you time, consider donating 5,- EUR! :-)
%%%%%%%%%%%%%%%%%%%%%%%%%%%%%%%%%%%%%%%%%%%%%%%%%%%%%%%%%%%%%%%%%%%%%%
%\annotatedFigureBoxCustom{bottom-left}{top-right}{label}{label-position}{box-color}{label-color}{border-color}{text-color}
\newcommand*\annotatedFigureBoxCustom[8]{\draw[#5,thick,rounded corners] (#1) rectangle (#2);\node at (#4) [fill=#6,thick,shape=circle,draw=#7,inner sep=2pt,font=\sffamily,text=#8] {\textbf{#3}};}
%\annotatedFigureBox{bottom-left}{top-right}{label}{label-position}
\newcommand*\annotatedFigureBox[4]{\annotatedFigureBoxCustom{#1}{#2}{#3}{#4}{white}{white}{black}{black}}
\newcommand*\annotatedFigureText[4]{\node[draw=none, anchor=south west, text=#2, inner sep=0, text width=#3\linewidth,font=\sffamily] at (#1){#4};}
\newenvironment {annotatedFigure}[1]{\centering\begin{tikzpicture}
\node[anchor=south west,inner sep=0] (image) at (0,0) { #1};\begin{scope}[x={(image.south east)},y={(image.north west)}]}{\end{scope}\end{tikzpicture}}
%%%%%%%%%%%%%%%%%%%%%%%%%%%%%%%%%%%%%%%%%%%%%%%%%%%%%%%%%%%%%%%%%%%%%%


\graphicspath{ {./report-images/} }
\begin{document}

\title{Electronics Online Challenge}
\author{Raider Robotics - MSOE1}
\submissionDate{12/7/2020}

\maketitle

\tableofcontents

\newpage

\section{Introduction}
For this challenge, we chose to disassemble a Segway i2 SE PT. One of our teammates obtained 
the Segway in non-working condition and is currently trying to rebuild the electronics to add
features to the device. We decided that this challenge would provide an excellent opportunity for
us to gain a much better understanding to the proprietary systems inside of the device, and understand
the engineering that makes the devices as robust as they are.

\section{Component Summary}

\begin{center}

% Please add the following required packages to your document preamble:
% \usepackage[table,xcdraw]{xcolor}
% If you use beamer only pass "xcolor=table" option, i.e. \documentclass[xcolor=table]{beamer}
% \usepackage[normalem]{ulem}
% \useunder{\uline}{\ul}{}
    % Please add the following required packages to your document preamble:
% \usepackage[normalem]{ulem}
% \useunder{\uline}{\ul}{}
\begin{table}[]
    \begin{tabular}{|l|l|l|l|l|l|}
    \hline
    Ref \# & Chip Identifier                          & Description                                & Documentation & Quantity \\ \hline
    1                & 16126797 AU195 02               &                                            & Not Found              & 1        \\ \hline
    2                & 24C04WQ K525W                            & I2C Bus EEPROM                             & Actual Component        & 2        \\ \hline
    3                & 37021 58M C66L                           &                                            & Not Found               & 1        \\ \hline
    4                & 37021 58MCD29                            &                                            & Not Found               & 2        \\ \hline
    5                & 431AV PAHF                               &                                            & Not Found               & 1        \\ \hline
    6                & 55 84 K0                                 &                                            & Not Found               & 1        \\ \hline
    7                & 56A504M HCT04                            & Hex Inverter                               & Similar Component       & 2        \\ \hline
    8                & 56A5NLM HCT4051M                         & Demultiplexer                   & Actual Component        & 3        \\ \hline
    9                & 59A2VHM TLC2254                          & Op Amp              & Actual Component        & 2        \\ \hline
    10               & 66C1HCM HCT4053M                         & Multiplexer  & Similar Component       & 1        \\ \hline
    11               & 7438 543C G68V                           &                                            & Not Found               & 1        \\ \hline
    12               & 8L05A POIB8                              & Positive Voltage Regulator                 & Similar Component       & 2        \\ \hline
    13               & A 7840 0611                              & Isolation Amplifier                        & Similar Component       & 4        \\ \hline
    14               & A82C250 4R4T0 n5064                      &                                            & Not Found               & 1        \\ \hline
    15               & BL05A POIB8                              &                                            & Not Found               & 2        \\ \hline
    16               & CHAQ LMC64 82AIM                         & Operational Amplifier & Actual Component        & 1        \\ \hline
    17               & CRLNLMG1 32B1M                           &                                            & Not Found             & 3        \\ \hline
    18               & IR 2136S 0515                            & Three-phase MOSFET Driver             & Actual Component        & 2        \\ \hline
    19               & IRFP250N G3 DB                           & N-Channel Power MOSFET                     & Actual Component        & 12       \\ \hline
    20               & K0204 FQA 70N15                          & N-Channel QFET MOSFET                      & Actual Component        & 1        \\ \hline
    21               & K537 TOP414G 35721A                      & DC/DC PWM Switch                           & Actual Component        & 1        \\ \hline
    22               & P185B MM74HCT 244WM                      & Octal 3-State Buffer                       & Actual Component        & 1        \\ \hline
    23               & P56AB 98752                              &                                            & Not Found               & 1        \\ \hline
    24               & PVI 1050 NS 0538I4N                      & Photovoltaic Isolator          & Actual Component       & 1        \\ \hline
    25               & TMS 320LF2406APZA        & DSP Controller                             & Not Found              & 1        \\ \hline
    \end{tabular}
    \end{table}
\end{center}

\section{Findings}
The Segway contains two main processing boards, which are identical mirrors. The processing board was
originally covered in a waterproof conformal coating to protect the electronics, which we removed selectively.
We found an array of chips and other components, and will attempt to classify them according to 
three different states of background information:
\begin{enumerate}
    \item \textbf{Fully Classified:} Data for the exact component was obtained
    \item \textbf{Partly Classified:} Data for another components of the same package type and 
    similar ID were obtained
    \item \textbf{Not Classified:} No data for the given component could be obtained
\end{enumerate}

\section{Conclusion}


\end{document}